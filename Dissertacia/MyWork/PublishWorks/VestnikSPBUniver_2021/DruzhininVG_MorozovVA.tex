\documentclass[10pt]{article}
\usepackage[utf8]{inputenc}

\usepackage{amsfonts}
\usepackage{amsmath}
\usepackage[T2A]{fontenc}
\usepackage[english,russian]{babel}

%\usepackage[dvips]{graphicx}
%\DeclareGraphicsExtensions{.pdf,.png,.jpg}
%\usepackage[dvips]{color} english,

\usepackage{graphicx}
\graphicspath{{Pic/}}
\DeclareGraphicsExtensions{.pdf,.png,.jpg}

\usepackage{amsmath}

\textwidth=14 cm
\textheight=20 cm

\begin{document}

\large \noindent  УДК 51-72, 51-74, 519.688

\bigskip

 В. Г. Дружинин, В. А. Морозов
 
\begin{center}
\bf ТРЕХМЕРНАЯ МОДЕЛЬ, ОПИСЫВАЮЩАЯ ОТКЛОНЕНИЯ АСИММЕТРИЧНОЙ ИГЛЫ ПРИ ДВИЖЕНИИ В МЯГКИХ ТКАНЯХ
\end{center}

\bigskip
\textbf{1. Введение}

\bigskip
В наши дни роботы все больше заменяют ручной труд человека. Машины уже могут выполнять не только монотонные производственные действия, но и заменять человека в более сложных операциях. К примеру можно отнести выполнение медицинских операций, как мало инвазивных, так и полноценных операций.
В настоящей статье рассматриваеться проведение операций брахитерапии рака предстательной железы (РПЖ). На сегодняшний день в ЦНИИ РТК разработан макет роботизированной системы «ОнкоРОБОТ») для проведения таких операций [1, 2]. 
Данная процедура проводится посредством внедрения микроисточников радиоизлучения в предстательную железу максимально близко к опухоли. Основная сложность заключается в подведения кончика иглы к целевой точке (опухоли) при проведении операции.
Преимущества использования роботов по сравнению с традиционными методами заключаются в том, что роботизированный манипулятор способен обеспечить более высокую точность наведения инструмента чем человек, а также контролировать силовое воздействие, что позволяет рассчитывать не только на повышение качества освоенных в настоящее время операций, но и на создание базиса для разработки принципиально новых хирургических технологий. 
Другим важным преимуществом является отсутствие прямого контакта врача с радиоактивными источниками, что позволит обезопасить медицинский персонал от сопутствующего радиационного облучения.
Из-за своих геометрических особенностей и прилагаемых нагрузок в процессе выполнения операции игла деформируется, что приводит к отклонению иглы от прямолинейного движения. 
В работе [3] приведен разбор этапов разработки данной модели приведен анализ существующих методов и подходов, представлена двухмерная модель, описывающая отклонение иглы при движении в тканях человека. 
В данной работе будет решена задача позиционирования иглы в системе координат Оxyz, а также рассмотрены способы повышения точности модели. 
Управление движением иглы осуществляется путем поворота иглы вокруг своей оси. При этом кончик иглы поворачивается, а вместе с ним и плоскость изгиба дуги, 
и тем самым изменяется направление дальнейшего движения. При введении иглы вдоль прямолинейной траектории ее необходимо постоянно поворачивать.
Разработанную модель можно использовать для построения “MPC-регуляторов” – систем, работающих на основе предсказывающих моделей (Model predictive control).
К примеру, в статье [4] показан ход разработки такой системы, только подход для проектирования модели использовался иной. В [4] авторы использовали уравнение Лагранжа для определения положения кончика иглы. Также условия эксперимента и сама игла значительно отличались от тех, которые были рассмотрены в данной статье.

\bigskip
\textbf{2. Модель}

\bigskip
\textbf{2.1 Двумерная модель}

\bigskip
Таким образом необходимо разработать модель, которая позволяет прогнозировать и корректировать движение иглы в тканях человека.  В данной работе в качестве моделируемого объекта рассматривается стальная медицинская инъекционная игла длиной 100 мм, диаметром 1 мм с различными углами кончика (рис 1).
Для построения модели рассмотрим  уравнение равновесия сил при движении иглы ~\cite{Model}: 

\begin{equation} \label{eq1}
\Vec{F}_{needle} = \Vec{F_{t}} + \Vec{F_{f}} + \Vec{w}(x),
\end{equation}
где $\Vec{F_{t}}$ --- сила, действующая на кончик иглы, $\Vec{F_{f}}$ --- сила трения, $\Vec{w}(x)$ --- распределенная нагрузка, $\Vec{F}_{needle}$ --- сила с которой внедряется игла.

\begin{figure}[h]
\center{\includegraphics[scale=0.8]{n1.jpg}}
\caption{Форма используемой иглы, F – сила реакции среды}
\label{fig:n1}
\end{figure}

В данной работе будет рассмотрена  задача в следующей постановке:

\begin{equation} \label{eq2}
\Vec{F}_{needle} = \Vec{F_{t}}.
\end{equation}
Для решения поставленной задачи отклонение кончика и угол отклонения будем рассчитывать по формулам ~\cite{Model}:

\begin{equation} \label{eq3}
y_{n} = Fl(t)^3 / 2EJ_{x},
\end{equation}

\begin{equation} \label{eq4}
\theta = Fl(t)^2 / 2EJ_{x},
\end{equation}
где $n$ --- текущая итерация моделирования, $y_{n}$ --- отклонение кончика иглы, на текущем шаге времени,
$F$ --- сила действующая на кончик иглы, $J_{x}$ --- осевой момент инерции, 
$l(t)$ --- длина иглы, находящаяся в тканях человека, $t$ --- время, $E$ --- модуль Юнга, $\theta$ --- угол отклонения.

Для моделирования внешней силы F при перемещении иглы в тканях человека будет использована сила лобового сопротивления:

\begin{equation} \label{eq5}
F = C \rho v^2 S /2, 
\end{equation}
где $C$ --- коэффициент сопротивления, $\rho$ --- плотность, $v$ --- скорость перемещения иглы, $S$ --- характерная площадь тела, $S = V^{(2/3)}$, где $V$- объем тела.

Для расчета смещения иглы по выражениям \eqref{eq3} и \eqref{eq4} необходимо учитывать проекцию силы F на ось иглы.
В данной постановке задачи по предложенным выражениям \eqref{eq3}, \eqref{eq4}, \eqref{eq5}, будем рассчитывать отклонение итерационно, суммируя его с предыдущими шагами. Тем самым будет сохраняться отклонение на каждом шаге моделирования:

\begin{equation} \label{eq6}
y_{all} = \sum\limits_{1}^{n-1} y_{n},
\end{equation}
где $n$ --- текущая итерация моделирования, $y_{all}$ --- суммарное отклонение иглы при ее движении в тканях человека, $y_{n}$ --- отклонение кончика иглы, на текущем шаге времени.


\bigskip
\textbf{2.2 Трехмерная модель}

Для трехмерной модели используем систему координат, представленную на рисунке 2. В данном случае углом поворота будет считаться величина, на которую повернётся плоскость среза иглы.

\begin{figure}[h]
\center{\includegraphics[scale=0.5]{n2.jpg}}
\caption{Рассматриваемая система координат}
\label{fig:n2}
\end{figure}

Для расчета координат положения кончика иглы воспользуемся следующими выражениями:

\begin{equation} \label{eq7}
z_{n} = 
 \begin{cases}
   y_{n} sin(\gamma) &{ 0 \leq \gamma \leq \pi/2 }\\
   y_{n} sin(\pi - \gamma) &{ \pi/2 \leq \gamma \leq \pi }\\
  - y_{n} sin( \gamma - \pi) &{ \pi \leq \gamma \leq 3\pi/4 }\\
   - y_{n} sin(2\pi -  \gamma) &{ 3\pi/4 \leq \gamma \leq 2\pi }\\
 \end{cases}
\end{equation}

\begin{equation} \label{eq8}
z_{all} = \sum\limits_{1}^{n} z_{n},
\end{equation}

\begin{equation} \label{eq9}
y_{k} = 
 \begin{cases}
   y_{n} cos(\gamma) &{ 0 \leq \gamma \leq \pi/2 }\\
   - y_{n} cos(\pi - \gamma) &{ \pi/2 \leq \gamma \leq \pi }\\
  - y_{n} cos( \gamma - \pi) &{ \pi \leq \gamma \leq 3\pi/4 }\\
    y_{n} cos(2\pi -  \gamma) &{ 3\pi/4 \leq \gamma \leq 2\pi }\\
 \end{cases}
\end{equation}

\begin{equation} \label{eq10}
y_{all} = \sum\limits_{1}^{k} y_{k},
\end{equation}
где $\gamma$ --- угол, на который повернулась игла за время моделирования, $z_{all} $ --- компонента отклонения по оси Oz, $y_{all}$ --- компонента отклонения по оси Oy, $y_{n}$ --- отклонение за 1 такт выполнения модели.

Для расчета отклонения от оси Ox воспользуемся следующими выражением:
\begin{equation} \label{eq11}
d_{all} = \sqrt{y_{all}^2  +  z_{all} ^2},
\end{equation}
где $d$ --- общее отклонение кончика игла от оси Ox.

Таким образом, на каждом шаге моделирования будет анализироваться угол, на который повернулась игла. Затем будет вычисляться отклонение на данном шаге и переводиться в координаты. А из данных значений координат y и z вычисляется общее отклонение от оси Ox.
Далее рассмотрены результаты моделирования и приведено сравнение с экспериментальными данными.

\bigskip
\textbf{2.3 Расчет коэффициентов сопротивления}

Решаемая задача является многопараметрической и зависит от нескольких переменных, а именно от поступательной и вращательной скоростей движения иглы в тканях. Необходимо найти такое решение чтобы различие между экспериментальными и расчетными данными было минимальным. 
Результаты моделирования по двухмерной модели с постоянным коэффициентом C, взятым из справочника, показали достаточно большие погрешности. 

Исходя из этого, данный коэффициент будем представлять в виде некоторой функциональной зависимости от скорости перемещения иглы, построенной на основе экспериментальных данных.  Данный подход позволил  обеспечить минимальные 

Далее рассмотрим коэффициенты для различных вращательных скоростей 0, 3, 4, 5 рад/сек.  

Для вращательной скорости равной 0 рад/сек:
\begin{multline} \label{eq12}
C= 2.2293\cdot10^{11} v^6 - 2.5517\cdot10^{10} v^5+1.788\cdot10^9 v^4 - \\ -2.8053\cdot10^7 v^3 +3.6420\cdot10^5 v^2-2.4583\cdot10^3 v+7.4299.
\end{multline}
Для вращательной скорости равной 3 рад/сек:
\begin{multline} \label{eq13}
C= -6.1243\cdot10^{18} v^9 + 1.0095\cdot10^{18}v^8 -  7.2393\cdot10^{16} v^7 +\\+ 2.9601\cdot10^{15} v^6 - 7.5961\cdot10^{13} v^5 + 1.2673\cdot10^{12} v^4 - \\-1.3740\cdot10^{10} v^3 + 9.3490\cdot10^7 v^2 - 3.6459\cdot10^5 v+ 634.2858.
\end{multline}
Для вращательной скорости равной 4 рад/сек:
\begin{multline} \label{eq14}
C= -5.5744\cdot10^{18} v^9 + 9.29\cdot10^{17}v^8 -  6.7439\cdot10^{16} v^7 +\\+ 2.7959\cdot10^{13} v^6 - 7.2889\cdot10^{13} v^5 + 1.2385\cdot10^{12} v^4 - \\-1.3720\cdot10^{10} v^3 + 9.5768\cdot10^7 v^2 - 3.3892\cdot10^5 v+ 693.0468.
\end{multline}
Для вращательной скорости равной 5 рад/сек:
\begin{multline} \label{eq14}
C= -1.5127\cdot10^{19} v^9 + 2.4827\cdot10^{18}v^8 -  1.7730\cdot10^{17} v^7 +\\+ 7.2242\cdot10^{15} v^6 - 1.8491\cdot10^{14} v^5 + 3.082\cdot10^{12} v^4 - \\-3.3464\cdot10^{10} v^3 + 2.2881\cdot10^7 v^2 - 9.0014\cdot10^5 v+ 1.5829\cdot10^3.
\end{multline}


\begin{tabular}{ | c | c | c | c | c | }
\hline
 & \multicolumn{4}{c|}{Коэффициенты С}   \\ 
 \cline{2-3}
 \raisebox{1.5ex}[0cm][0cm] {Линейная скорость, мм/с} & 0 рад/с & 2 рад/с & 4 рад/с & 5 рад/с \\ \hline
3 & 2,664938602 & 97,13340693
6 & 1,07169768 & 15,73135406
9 & 0,700671524 & 6,365846027
12 & 0,659327251 & 3,23079564
15 & 0,660593963 & 1,842929895
18 & 0,688335976 & 1,204554851
21 & 0,779838013 &  0,795031504
24 & 0,925302352 & 0,559047066
27 & 1,084357932 & 0,403247032
30 & 1,319581413 & 0,312718332

\hline
\end{tabular}


\bigskip
\begin{thebibliography}{99}
\small

\bibitem{Model}{В.Г. Дружинин, В.А. Морозов, С.А. Никитин, В.В. Харламов.}{\; Модель Отклонения медицинской иглы при движении в тканях человека //Российский журнал биомеханики выпуск 4 2018 С 459-472.}

\end{thebibliography}

\end{document}

